\documentclass[12pt,a4paper]{article}
\usepackage[utf8]{inputenc}
\usepackage{amsmath}
\usepackage{amsfonts}
\usepackage{amssymb}
\usepackage{geometry}
\usepackage{xeCJK}
\usepackage{fontspec}
\usepackage{enumitem}
\usepackage{titlesec}
\usepackage{fancyhdr}
\usepackage{hyperref}
\usepackage{unicode-math}  % Better Unicode math support

% Set up Chinese fonts (requires XeLaTeX)
\setCJKmainfont{SimSun}

% Set paragraph indentation for Chinese text
\setlength{\parindent}{2em}  % 设置段落首行缩进为2个字符宽度

% Page setup
\geometry{margin=2.5cm}
\pagestyle{fancy}
\fancyhf{}
\fancyhead[C]{经典数学公式集 (Classic Mathematical Formulas)}
\fancyfoot[C]{\thepage}

% Title formatting
\titleformat{\section}{\Large\bfseries}{\thesection}{1em}{}
\titleformat{\subsection}{\large\bfseries}{\thesubsection}{1em}{}

% Hyperref setup
\hypersetup{
    colorlinks=true,
    linkcolor=blue,
    filecolor=magenta,      
    urlcolor=cyan,
    pdftitle={Classic Mathematical Formulas},
    pdfauthor={Auto-generated},
    pdfsubject={Mathematics},
    pdfkeywords={mathematics, formulas, LaTeX}
}

\begin{document}

% Title page
\begin{titlepage}
\centering
\vspace*{2cm}
{\Huge\bfseries 经典数学公式集}\\[0.5cm]
{\Large\bfseries Classic Mathematical Formulas}\\[2cm]
{\large 包含中英文对照的数学公式汇编}\\[0.3cm]
{\large A Collection of Mathematical Formulas with Chinese and English Descriptions}\\[3cm]
{\large 编译日期 (Compiled Date): \today}\\[2cm]
\vfill
{\large 本文档由Python脚本自动生成}\\
{\large This document was automatically generated by a Python script}
\end{titlepage}

\newpage

\section{数学公式 (Mathematical Formulas)}

\begin{enumerate}[leftmargin=*]

\item \textbf{公式 1 (Formula 1)}

\begin{equation}
a + b = b + a
\end{equation}

\textbf{中文说明:} 加法交换律

\textbf{English Description:} Commutative law of addition

\vspace{0.5cm}

\item \textbf{公式 2 (Formula 2)}

\begin{equation}
a \times b = b \times a
\end{equation}

\textbf{中文说明:} 乘法交换律

\textbf{English Description:} Commutative law of multiplication

\vspace{0.5cm}

\item \textbf{公式 3 (Formula 3)}

\begin{equation}
(a + b) + c = a + (b + c)
\end{equation}

\textbf{中文说明:} 加法结合律

\textbf{English Description:} Associative law of addition

\vspace{0.5cm}

\item \textbf{公式 4 (Formula 4)}

\begin{equation}
(a \times b) \times c = a \times (b \times c)
\end{equation}

\textbf{中文说明:} 乘法结合律

\textbf{English Description:} Associative law of multiplication

\vspace{0.5cm}

\item \textbf{公式 5 (Formula 5)}

\begin{equation}
a \times (b + c) = a \times b + a \times c
\end{equation}

\textbf{中文说明:} 乘法对加法的分配律

\textbf{English Description:} Distributive law of multiplication over addition

\vspace{0.5cm}

\item \textbf{公式 6 (Formula 6)}

\begin{equation}
a - b = a + (-b)
\end{equation}

\textbf{中文说明:} 减法的定义

\textbf{English Description:} Definition of subtraction

\vspace{0.5cm}

\item \textbf{公式 7 (Formula 7)}

\begin{equation}
a \div b = a \times \frac{1}{b}\ (b \neq 0)
\end{equation}

\textbf{中文说明:} 除法的定义

\textbf{English Description:} Definition of division

\vspace{0.5cm}

\item \textbf{公式 8 (Formula 8)}

\begin{equation}
S = a \times b
\end{equation}

\textbf{中文说明:} 长方形的面积

\textbf{English Description:} Area of a rectangle

\vspace{0.5cm}

\item \textbf{公式 9 (Formula 9)}

\begin{equation}
S = \frac{1}{2} a \times h
\end{equation}

\textbf{中文说明:} 三角形的面积

\textbf{English Description:} Area of a triangle

\vspace{0.5cm}

\item \textbf{公式 10 (Formula 10)}

\begin{equation}
C = 2\pi r
\end{equation}

\textbf{中文说明:} 圆的周长公式

\textbf{English Description:} Circumference of a circle

\vspace{0.5cm}

\item \textbf{公式 11 (Formula 11)}

\begin{equation}
S = \pi r^2
\end{equation}

\textbf{中文说明:} 圆的面积公式

\textbf{English Description:} Area of a circle

\vspace{0.5cm}

\item \textbf{公式 12 (Formula 12)}

\begin{equation}
S = a \times a = a^2
\end{equation}

\textbf{中文说明:} 正方形的面积

\textbf{English Description:} Area of a square

\vspace{0.5cm}

\item \textbf{公式 13 (Formula 13)}

\begin{equation}
P = 4a
\end{equation}

\textbf{中文说明:} 正方形的周长

\textbf{English Description:} Perimeter of a square

\vspace{0.5cm}

\item \textbf{公式 14 (Formula 14)}

\begin{equation}
P = 2(a + b)
\end{equation}

\textbf{中文说明:} 长方形的周长

\textbf{English Description:} Perimeter of a rectangle

\vspace{0.5cm}

\item \textbf{公式 15 (Formula 15)}

\begin{equation}
S = \frac{(a + b)h}{2}
\end{equation}

\textbf{中文说明:} 梯形的面积

\textbf{English Description:} Area of a trapezoid

\vspace{0.5cm}

\item \textbf{公式 16 (Formula 16)}

\begin{equation}
S = a \times h
\end{equation}

\textbf{中文说明:} 平行四边形的面积

\textbf{English Description:} Area of a parallelogram

\vspace{0.5cm}

\item \textbf{公式 17 (Formula 17)}

\begin{equation}
V = a^3
\end{equation}

\textbf{中文说明:} 正方体的体积

\textbf{English Description:} Volume of a cube

\vspace{0.5cm}

\item \textbf{公式 18 (Formula 18)}

\begin{equation}
x = \frac{-b \pm \sqrt{b^2-4ac}}{2a}
\end{equation}

\textbf{中文说明:} 一元二次方程求根公式

\textbf{English Description:} Formula for finding roots of quadratic equation

\vspace{0.5cm}

\item \textbf{公式 19 (Formula 19)}

\begin{equation}
a^2 + b^2 = c^2
\end{equation}

\textbf{中文说明:} 勾股定理

\textbf{English Description:} Pythagorean theorem

\vspace{0.5cm}

\item \textbf{公式 20 (Formula 20)}

\begin{equation}
\int_a^b f(x)dx = F(b) - F(a)
\end{equation}

\textbf{中文说明:} 微积分基本定理

\textbf{English Description:} Fundamental theorem of calculus

\vspace{0.5cm}

\item \textbf{公式 21 (Formula 21)}

\begin{equation}
a_n = a_1 + (n-1)d
\end{equation}

\textbf{中文说明:} 等差数列第n项公式

\textbf{English Description:} Nth term of an arithmetic sequence

\vspace{0.5cm}

\item \textbf{公式 22 (Formula 22)}

\begin{equation}
S_n = \frac{n}{2}(a_1 + a_n)
\end{equation}

\textbf{中文说明:} 等差数列前n项和公式

\textbf{English Description:} Sum of the first n terms of an arithmetic sequence

\vspace{0.5cm}

\item \textbf{公式 23 (Formula 23)}

\begin{equation}
a_n = a_1 r^{n-1}
\end{equation}

\textbf{中文说明:} 等比数列第n项公式

\textbf{English Description:} Nth term of a geometric sequence

\vspace{0.5cm}

\item \textbf{公式 24 (Formula 24)}

\begin{equation}
S_n = a_1 \frac{1-r^n}{1-r}
\end{equation}

\textbf{中文说明:} 等比数列前n项和公式(r $\neq$ 1)

\textbf{English Description:} Sum of the first n terms of a geometric sequence (r $\neq$ 1)

\vspace{0.5cm}

\item \textbf{公式 25 (Formula 25)}

\begin{equation}
\log_a{b} = \frac{\ln{b}}{\ln{a}}
\end{equation}

\textbf{中文说明:} 对数换底公式

\textbf{English Description:} Change of base formula for logarithms

\vspace{0.5cm}

\item \textbf{公式 26 (Formula 26)}

\begin{equation}
\binom{n}{k} = \frac{n!}{k!(n-k)!}
\end{equation}

\textbf{中文说明:} 组合数公式

\textbf{English Description:} Number of combinations (n choose k)

\vspace{0.5cm}

\item \textbf{公式 27 (Formula 27)}

\begin{equation}
e^x = \sum_{n=0}^{\infty} \frac{x^n}{n!}
\end{equation}

\textbf{中文说明:} 指数函数的泰勒展开式

\textbf{English Description:} Taylor expansion for exponential function

\vspace{0.5cm}

\item \textbf{公式 28 (Formula 28)}

\begin{equation}
\sin^2 x + \cos^2 x = 1
\end{equation}

\textbf{中文说明:} 三角恒等式

\textbf{English Description:} Pythagorean trigonometric identity

\vspace{0.5cm}

\item \textbf{公式 29 (Formula 29)}

\begin{equation}
\sin(a \pm b) = \sin a \cos b \pm \cos a \sin b
\end{equation}

\textbf{中文说明:} 正弦加减法公式

\textbf{English Description:} Sine addition and subtraction formulas

\vspace{0.5cm}

\item \textbf{公式 30 (Formula 30)}

\begin{equation}
\cos(a \pm b) = \cos a \cos b \mp \sin a \sin b
\end{equation}

\textbf{中文说明:} 余弦加减法公式

\textbf{English Description:} Cosine addition and subtraction formulas

\vspace{0.5cm}

\item \textbf{公式 31 (Formula 31)}

\begin{equation}
\tan(a \pm b) = \frac{\tan a \pm \tan b}{1 \mp \tan a \tan b}
\end{equation}

\textbf{中文说明:} 正切加减法公式

\textbf{English Description:} Tangent addition and subtraction formulas

\vspace{0.5cm}

\item \textbf{公式 32 (Formula 32)}

\begin{equation}
\frac{d}{dx}x^n = nx^{n-1}
\end{equation}

\textbf{中文说明:} 幂函数求导法则

\textbf{English Description:} Power rule for differentiation

\vspace{0.5cm}

\item \textbf{公式 33 (Formula 33)}

\begin{equation}
\frac{d}{dx}\sin x = \cos x
\end{equation}

\textbf{中文说明:} 正弦函数的导数

\textbf{English Description:} Derivative of sine function

\vspace{0.5cm}

\item \textbf{公式 34 (Formula 34)}

\begin{equation}
\frac{d}{dx}\cos x = -\sin x
\end{equation}

\textbf{中文说明:} 余弦函数的导数

\textbf{English Description:} Derivative of cosine function

\vspace{0.5cm}

\item \textbf{公式 35 (Formula 35)}

\begin{equation}
\frac{d}{dx}e^x = e^x
\end{equation}

\textbf{中文说明:} 指数函数的导数

\textbf{English Description:} Derivative of exponential function

\vspace{0.5cm}

\item \textbf{公式 36 (Formula 36)}

\begin{equation}
\frac{d}{dx}\ln x = \frac{1}{x}
\end{equation}

\textbf{中文说明:} 自然对数的导数

\textbf{English Description:} Derivative of natural logarithm

\vspace{0.5cm}

\item \textbf{公式 37 (Formula 37)}

\begin{equation}
\int x^n dx = \frac{x^{n+1}}{n+1} + C\ (n \neq -1)
\end{equation}

\textbf{中文说明:} 幂函数的积分

\textbf{English Description:} Integral of power function

\vspace{0.5cm}

\item \textbf{公式 38 (Formula 38)}

\begin{equation}
\int e^{ax} dx = \frac{1}{a}e^{ax} + C
\end{equation}

\textbf{中文说明:} 指数函数的积分

\textbf{English Description:} Integral of exponential function

\vspace{0.5cm}

\item \textbf{公式 39 (Formula 39)}

\begin{equation}
\int \sin x dx = -\cos x + C
\end{equation}

\textbf{中文说明:} 正弦函数的积分

\textbf{English Description:} Integral of sine function

\vspace{0.5cm}

\item \textbf{公式 40 (Formula 40)}

\begin{equation}
\int \cos x dx = \sin x + C
\end{equation}

\textbf{中文说明:} 余弦函数的积分

\textbf{English Description:} Integral of cosine function

\vspace{0.5cm}

\item \textbf{公式 41 (Formula 41)}

\begin{equation}
\sum_{k=1}^{n} k = \frac{n(n+1)}{2}
\end{equation}

\textbf{中文说明:} 前n个正整数的和

\textbf{English Description:} Sum of the first n positive integers

\vspace{0.5cm}

\item \textbf{公式 42 (Formula 42)}

\begin{equation}
\sum_{k=1}^{n} k^2 = \frac{n(n+1)(2n+1)}{6}
\end{equation}

\textbf{中文说明:} 前n个正整数的平方和

\textbf{English Description:} Sum of the squares of the first n positive integers

\vspace{0.5cm}

\item \textbf{公式 43 (Formula 43)}

\begin{equation}
\sum_{k=1}^{n} k^3 = \left[\frac{n(n+1)}{2}\right]^2
\end{equation}

\textbf{中文说明:} 前n个正整数的立方和

\textbf{English Description:} Sum of the cubes of the first n positive integers

\vspace{0.5cm}

\item \textbf{公式 44 (Formula 44)}

\begin{equation}
a^m \cdot a^n = a^{m+n}
\end{equation}

\textbf{中文说明:} 同底数幂的乘法

\textbf{English Description:} Product of powers with same base

\vspace{0.5cm}

\item \textbf{公式 45 (Formula 45)}

\begin{equation}
(a^m)^n = a^{mn}
\end{equation}

\textbf{中文说明:} 幂的幂

\textbf{English Description:} Power of a power

\vspace{0.5cm}

\item \textbf{公式 46 (Formula 46)}

\begin{equation}
(ab)^n = a^n b^n
\end{equation}

\textbf{中文说明:} 积的幂

\textbf{English Description:} Power of a product

\vspace{0.5cm}

\item \textbf{公式 47 (Formula 47)}

\begin{equation}
\sqrt[n]{a} = a^{1/n}
\end{equation}

\textbf{中文说明:} n次方根的定义

\textbf{English Description:} Definition of n-th root

\vspace{0.5cm}

\item \textbf{公式 48 (Formula 48)}

\begin{equation}
\log_a{1} = 0
\end{equation}

\textbf{中文说明:} 1的对数为0

\textbf{English Description:} Logarithm of 1 is 0

\vspace{0.5cm}

\item \textbf{公式 49 (Formula 49)}

\begin{equation}
\log_a{a} = 1
\end{equation}

\textbf{中文说明:} 以自身为底的对数为1

\textbf{English Description:} Logarithm of base is 1

\vspace{0.5cm}

\item \textbf{公式 50 (Formula 50)}

\begin{equation}
\log_a{bc} = \log_a{b} + \log_a{c}
\end{equation}

\textbf{中文说明:} 对数的乘法公式

\textbf{English Description:} Logarithm of a product

\vspace{0.5cm}

\item \textbf{公式 51 (Formula 51)}

\begin{equation}
\log_a{\frac{b}{c}} = \log_a{b} - \log_a{c}
\end{equation}

\textbf{中文说明:} 对数的除法公式

\textbf{English Description:} Logarithm of a quotient

\vspace{0.5cm}

\item \textbf{公式 52 (Formula 52)}

\begin{equation}
\log_a{b^n} = n \log_a{b}
\end{equation}

\textbf{中文说明:} 幂的对数

\textbf{English Description:} Logarithm of a power

\vspace{0.5cm}

\item \textbf{公式 53 (Formula 53)}

\begin{equation}
\lim_{x \to 0} \frac{\sin x}{x} = 1
\end{equation}

\textbf{中文说明:} sinx/x在x趋近于0时的极限

\textbf{English Description:} Limit of sin x over x as x approaches 0

\vspace{0.5cm}

\item \textbf{公式 54 (Formula 54)}

\begin{equation}
\lim_{n \to \infty} \left(1 + \frac{1}{n}\right)^n = e
\end{equation}

\textbf{中文说明:} 欧拉数的定义

\textbf{English Description:} Definition of Euler's number

\vspace{0.5cm}

\item \textbf{公式 55 (Formula 55)}

\begin{equation}
|a+b| \leq |a| + |b|
\end{equation}

\textbf{中文说明:} 三角不等式

\textbf{English Description:} Triangle inequality

\vspace{0.5cm}

\item \textbf{公式 56 (Formula 56)}

\begin{equation}
\frac{a}{b} = \frac{c}{d} \implies ad = bc
\end{equation}

\textbf{中文说明:} 比例的性质(交叉相乘)

\textbf{English Description:} Proportion property (cross multiplication)

\vspace{0.5cm}

\item \textbf{公式 57 (Formula 57)}

\begin{equation}
(a+b)^2 = a^2 + 2ab + b^2
\end{equation}

\textbf{中文说明:} 二项式的平方(加法)

\textbf{English Description:} Square of a binomial (addition)

\vspace{0.5cm}

\item \textbf{公式 58 (Formula 58)}

\begin{equation}
(a-b)^2 = a^2 - 2ab + b^2
\end{equation}

\textbf{中文说明:} 二项式的平方(减法)

\textbf{English Description:} Square of a binomial (subtraction)

\vspace{0.5cm}

\item \textbf{公式 59 (Formula 59)}

\begin{equation}
(a+b)(a-b) = a^2 - b^2
\end{equation}

\textbf{中文说明:} 和与差的积

\textbf{English Description:} Product of sum and difference

\vspace{0.5cm}

\item \textbf{公式 60 (Formula 60)}

\begin{equation}
a^3 + b^3 = (a + b)(a^2 - ab + b^2)
\end{equation}

\textbf{中文说明:} 立方和公式

\textbf{English Description:} Sum of cubes

\vspace{0.5cm}

\item \textbf{公式 61 (Formula 61)}

\begin{equation}
a^3 - b^3 = (a - b)(a^2 + ab + b^2)
\end{equation}

\textbf{中文说明:} 立方差公式

\textbf{English Description:} Difference of cubes

\vspace{0.5cm}

\item \textbf{公式 62 (Formula 62)}

\begin{equation}
\sin 2x = 2 \sin x \cos x
\end{equation}

\textbf{中文说明:} 正弦的二倍角公式

\textbf{English Description:} Double angle formula for sine

\vspace{0.5cm}

\item \textbf{公式 63 (Formula 63)}

\begin{equation}
\cos 2x = \cos^2 x - \sin^2 x
\end{equation}

\textbf{中文说明:} 余弦的二倍角公式

\textbf{English Description:} Double angle formula for cosine

\vspace{0.5cm}

\item \textbf{公式 64 (Formula 64)}

\begin{equation}
\tan 2x = \frac{2\tan x}{1 - \tan^2 x}
\end{equation}

\textbf{中文说明:} 正切的二倍角公式

\textbf{English Description:} Double angle formula for tangent

\vspace{0.5cm}

\item \textbf{公式 65 (Formula 65)}

\begin{equation}
\sin^2 x = \frac{1 - \cos 2x}{2}
\end{equation}

\textbf{中文说明:} 正弦平方的降幂公式

\textbf{English Description:} Power reduction formula for sine squared

\vspace{0.5cm}

\item \textbf{公式 66 (Formula 66)}

\begin{equation}
\cos^2 x = \frac{1 + \cos 2x}{2}
\end{equation}

\textbf{中文说明:} 余弦平方的降幂公式

\textbf{English Description:} Power reduction formula for cosine squared

\vspace{0.5cm}

\item \textbf{公式 67 (Formula 67)}

\begin{equation}
\tan^2 x = \frac{1 - \cos 2x}{1 + \cos 2x}
\end{equation}

\textbf{中文说明:} 正切平方的降幂公式

\textbf{English Description:} Power reduction formula for tangent squared

\vspace{0.5cm}

\item \textbf{公式 68 (Formula 68)}

\begin{equation}
\tan(x+y) = \frac{\tan x + \tan y}{1 - \tan x \tan y}
\end{equation}

\textbf{中文说明:} 正切的和公式

\textbf{English Description:} Tangent of sum formula

\vspace{0.5cm}

\item \textbf{公式 69 (Formula 69)}

\begin{equation}
\frac{d}{dx}\tan x = \sec^2 x
\end{equation}

\textbf{中文说明:} 正切函数的导数

\textbf{English Description:} Derivative of tangent function

\vspace{0.5cm}

\item \textbf{公式 70 (Formula 70)}

\begin{equation}
\int \frac{1}{x} dx = \ln|x| + C
\end{equation}

\textbf{中文说明:} 1/x的积分

\textbf{English Description:} Integral of 1/x

\vspace{0.5cm}

\item \textbf{公式 71 (Formula 71)}

\begin{equation}
\lim_{n \to \infty}\left(1+\frac{x}{n}\right)^n = e^x
\end{equation}

\textbf{中文说明:} 指数极限定义

\textbf{English Description:} Exponential limit definition

\vspace{0.5cm}

\item \textbf{公式 72 (Formula 72)}

\begin{equation}
\lim_{x \to 0} \frac{1-\cos x}{x^2} = \frac{1}{2}
\end{equation}

\textbf{中文说明:} (1-cosx)/x\textasciicircum{}2在x趋近于0时的极限

\textbf{English Description:} Limit of (1-cosx)/x\textasciicircum{}2 as x approaches 0

\vspace{0.5cm}

\item \textbf{公式 73 (Formula 73)}

\begin{equation}
f(x) = f(a) + f'(a)(x-a) + \frac{f''(a)}{2!}(x-a)^2 + \cdots
\end{equation}

\textbf{中文说明:} 函数在x=a处的泰勒展开式

\textbf{English Description:} Taylor series expansion of f(x) at x=a

\vspace{0.5cm}

\item \textbf{公式 74 (Formula 74)}

\begin{equation}
f(x, y) = \frac{\partial f}{\partial x} dx + \frac{\partial f}{\partial y} dy
\end{equation}

\textbf{中文说明:} 二元函数的全微分

\textbf{English Description:} Total differential of a function of two variables

\vspace{0.5cm}

\item \textbf{公式 75 (Formula 75)}

\begin{equation}
\frac{\partial z}{\partial x}
\end{equation}

\textbf{中文说明:} z关于x的偏导数

\textbf{English Description:} Partial derivative of z with respect to x

\vspace{0.5cm}

\item \textbf{公式 76 (Formula 76)}

\begin{equation}
\frac{\partial^2 f}{\partial x^2}
\end{equation}

\textbf{中文说明:} f关于x的二阶偏导数

\textbf{English Description:} Second order partial derivative of f with respect to x

\vspace{0.5cm}

\item \textbf{公式 77 (Formula 77)}

\begin{equation}
\nabla f = \left(\frac{\partial f}{\partial x}, \frac{\partial f}{\partial y}, \frac{\partial f}{\partial z}\right)
\end{equation}

\textbf{中文说明:} 三元函数的梯度

\textbf{English Description:} Gradient of a scalar function in three dimensions

\vspace{0.5cm}

\item \textbf{公式 78 (Formula 78)}

\begin{equation}
\text{div}\, \vec{F} = \nabla \cdot \vec{F} = \frac{\partial F_1}{\partial x} + \frac{\partial F_2}{\partial y} + \frac{\partial F_3}{\partial z}
\end{equation}

\textbf{中文说明:} 三维向量场的散度

\textbf{English Description:} Divergence of a vector field in three dimensions

\vspace{0.5cm}

\item \textbf{公式 79 (Formula 79)}

\begin{equation}
\text{curl}\, \vec{F} = \nabla \times \vec{F}
\end{equation}

\textbf{中文说明:} 向量场的旋度

\textbf{English Description:} Curl of a vector field

\vspace{0.5cm}

\item \textbf{公式 80 (Formula 80)}

\begin{equation}
\Delta f = \nabla^2 f = \frac{\partial^2 f}{\partial x^2} + \frac{\partial^2 f}{\partial y^2} + \frac{\partial^2 f}{\partial z^2}
\end{equation}

\textbf{中文说明:} 三元函数的拉普拉斯算子

\textbf{English Description:} Laplacian of a function in three dimensions

\vspace{0.5cm}

\item \textbf{公式 81 (Formula 81)}

\begin{equation}
\iint_D f(x, y)\,dx\,dy
\end{equation}

\textbf{中文说明:} 区域D上的二重积分

\textbf{English Description:} Double integral over region D

\vspace{0.5cm}

\item \textbf{公式 82 (Formula 82)}

\begin{equation}
\iiint_E f(x, y, z)\,dx\,dy\,dz
\end{equation}

\textbf{中文说明:} 区域E上的三重积分

\textbf{English Description:} Triple integral over region E

\vspace{0.5cm}

\item \textbf{公式 83 (Formula 83)}

\begin{equation}
\frac{\partial u}{\partial t} = k \frac{\partial^2 u}{\partial x^2}
\end{equation}

\textbf{中文说明:} 一维热传导方程,经典偏微分方程

\textbf{English Description:} One-dimensional heat equation, a classic partial differential equation

\vspace{0.5cm}

\item \textbf{公式 84 (Formula 84)}

\begin{equation}
\frac{\partial^2 u}{\partial t^2} = c^2 \frac{\partial^2 u}{\partial x^2}
\end{equation}

\textbf{中文说明:} 一维波动方程

\textbf{English Description:} One-dimensional wave equation

\vspace{0.5cm}

\item \textbf{公式 85 (Formula 85)}

\begin{equation}
\frac{\partial^2 u}{\partial x^2} + \frac{\partial^2 u}{\partial y^2} = 0
\end{equation}

\textbf{中文说明:} 二维拉普拉斯方程

\textbf{English Description:} Laplace's equation in two variables

\vspace{0.5cm}

\item \textbf{公式 86 (Formula 86)}

\begin{equation}
\nabla^2 u = f(x, y, z)
\end{equation}

\textbf{中文说明:} 三维泊松方程

\textbf{English Description:} Poisson equation in three dimensions

\vspace{0.5cm}

\item \textbf{公式 87 (Formula 87)}

\begin{equation}
u(x, t) = \sum_{n=1}^\infty A_n \sin\left(\frac{n\pi x}{L}\right) e^{-k\left(\frac{n\pi}{L}\right)^2 t}
\end{equation}

\textbf{中文说明:} 热方程的变量分离法解(Dirichlet边界条件)

\textbf{English Description:} Separation of variables solution for the heat equation with Dirichlet boundary conditions

\vspace{0.5cm}

\item \textbf{公式 88 (Formula 88)}

\begin{equation}
\int_{C} \vec{F} \cdot d\vec{r}
\end{equation}

\textbf{中文说明:} 沿曲线C的向量场线积分

\textbf{English Description:} Line integral of a vector field along curve C

\vspace{0.5cm}

\item \textbf{公式 89 (Formula 89)}

\begin{equation}
\iint_{S} \vec{F} \cdot d\vec{S}
\end{equation}

\textbf{中文说明:} 向量场在曲面S上的面积分

\textbf{English Description:} Surface integral of a vector field over surface S

\vspace{0.5cm}

\item \textbf{公式 90 (Formula 90)}

\begin{equation}
\frac{\partial^2 u}{\partial x^2} + \frac{\partial^2 u}{\partial y^2} + \frac{\partial^2 u}{\partial z^2} = 0
\end{equation}

\textbf{中文说明:} 三维拉普拉斯方程

\textbf{English Description:} Laplace's equation in three variables

\vspace{0.5cm}

\item \textbf{公式 91 (Formula 91)}

\begin{equation}
\frac{\partial^2 u}{\partial t^2} - c^2 \nabla^2 u = 0
\end{equation}

\textbf{中文说明:} 多维波动方程

\textbf{English Description:} Wave equation in multiple dimensions

\vspace{0.5cm}

\item \textbf{公式 92 (Formula 92)}

\begin{equation}
u(x, y) = X(x)Y(y)
\end{equation}

\textbf{中文说明:} 偏微分方程的变量分离形式

\textbf{English Description:} Separation of variables form for PDEs

\vspace{0.5cm}

\item \textbf{公式 93 (Formula 93)}

\begin{equation}
\nabla \cdot (\nabla u) = \nabla^2 u
\end{equation}

\textbf{中文说明:} 拉普拉斯算子的定义

\textbf{English Description:} Definition of Laplacian operator

\vspace{0.5cm}

\item \textbf{公式 94 (Formula 94)}

\begin{equation}
\det(AB) = \det(A)\det(B)
\end{equation}

\textbf{中文说明:} 两个矩阵乘积的行列式

\textbf{English Description:} Determinant of the product of two matrices

\vspace{0.5cm}

\item \textbf{公式 95 (Formula 95)}

\begin{equation}
(A^{-1})^T = (A^T)^{-1}
\end{equation}

\textbf{中文说明:} 矩阵转置的逆

\textbf{English Description:} Inverse of the transpose of a matrix

\vspace{0.5cm}

\item \textbf{公式 96 (Formula 96)}

\begin{equation}
\det(A^T) = \det(A)
\end{equation}

\textbf{中文说明:} 矩阵转置的行列式

\textbf{English Description:} Determinant of the transpose of a matrix

\vspace{0.5cm}

\item \textbf{公式 97 (Formula 97)}

\begin{equation}
\det(A) = 0 \iff A\text{ is singular}
\end{equation}

\textbf{中文说明:} 行列式为0当且仅当矩阵奇异

\textbf{English Description:} A matrix is singular if and only if its determinant is zero

\vspace{0.5cm}

\item \textbf{公式 98 (Formula 98)}

\begin{equation}
\text{rank}(A) = \text{number of linearly independent rows (or columns)}
\end{equation}

\textbf{中文说明:} 矩阵秩的定义

\textbf{English Description:} Definition of matrix rank

\vspace{0.5cm}

\item \textbf{公式 99 (Formula 99)}

\begin{equation}
A\vec{x} = \lambda \vec{x}
\end{equation}

\textbf{中文说明:} 矩阵的特征值方程

\textbf{English Description:} Eigenvalue equation for matrix A

\vspace{0.5cm}

\item \textbf{公式 100 (Formula 100)}

\begin{equation}
\det\begin{pmatrix} a & b & c \\ d & e & f \\ g & h & i \end{pmatrix} = aei + bfg + cdh - ceg - bdi - afh
\end{equation}

\textbf{中文说明:} 三阶矩阵的行列式公式。

\textbf{English Description:} Determinant of a 3x3 matrix.

\vspace{0.5cm}

\item \textbf{公式 101 (Formula 101)}

\begin{equation}
\det\begin{pmatrix} a_{11} & a_{12} & a_{13} & a_{14} \\ a_{21} & a_{22} & a_{23} & a_{24} \\ a_{31} & a_{32} & a_{33} & a_{34} \\ a_{41} & a_{42} & a_{43} & a_{44} \end{pmatrix} = \sum_{j=1}^4 (-1)^{1+j} a_{1j} M_{1j}
\end{equation}

\textbf{中文说明:} 四阶矩阵按第一行展开的行列式公式。

\textbf{English Description:} Determinant of a 4x4 matrix by cofactor expansion along the first row.

\vspace{0.5cm}

\item \textbf{公式 102 (Formula 102)}

\begin{equation}
m\ddot{x} + c\dot{x} + kx = 0
\end{equation}

\textbf{中文说明:} 一般二阶线性阻尼运动方程(质量-弹簧-阻尼系统)。

\textbf{English Description:} General second-order linear damped motion equation (mass-spring-damper system).

\vspace{0.5cm}

\item \textbf{公式 103 (Formula 103)}

\begin{equation}
v_{n+1} = v_n + [ -2\zeta\omega_0 v_n - \omega_0^2 (x_n - x_{\text{target}}) ] \Delta t
\end{equation}

\textbf{中文说明:} 朝目标的阻尼弹簧数值模拟中常用的速度更新公式。

\textbf{English Description:} Discrete velocity update for critically/underdamped spring to a target (game-friendly).

\vspace{0.5cm}

\item \textbf{公式 104 (Formula 104)}

\begin{equation}
\sum_{i=1}^{n} a_i b_i
\end{equation}

\textbf{中文说明:} 两个向量的点积

\textbf{English Description:} Dot product of two vectors

\vspace{0.5cm}

\item \textbf{公式 105 (Formula 105)}

\begin{equation}
\vec{a} \cdot \vec{b} = |\vec{a}| |\vec{b}| \cos \theta
\end{equation}

\textbf{中文说明:} 点积的模和夹角表示

\textbf{English Description:} Dot product in terms of magnitude and angle

\vspace{0.5cm}

\item \textbf{公式 106 (Formula 106)}

\begin{equation}
\vec{a} \times \vec{b}
\end{equation}

\textbf{中文说明:} 两个向量的叉积

\textbf{English Description:} Cross product of two vectors

\vspace{0.5cm}

\item \textbf{公式 107 (Formula 107)}

\begin{equation}
\vec{a} \times \vec{b} = |\vec{a}| |\vec{b}| \sin \theta\ \vec{n}
\end{equation}

\textbf{中文说明:} 叉积的模和夹角表示

\textbf{English Description:} Cross product in terms of magnitude and angle

\vspace{0.5cm}

\item \textbf{公式 108 (Formula 108)}

\begin{equation}
\sqrt{a^2 + b^2}
\end{equation}

\textbf{中文说明:} 二维空间距离公式

\textbf{English Description:} Distance formula in 2D

\vspace{0.5cm}

\item \textbf{公式 109 (Formula 109)}

\begin{equation}
\sqrt{(x_2-x_1)^2 + (y_2-y_1)^2}
\end{equation}

\textbf{中文说明:} 两点间距离公式

\textbf{English Description:} Distance between two points (x1, y1) and (x2, y2)

\vspace{0.5cm}

\item \textbf{公式 110 (Formula 110)}

\begin{equation}
y = mx + b
\end{equation}

\textbf{中文说明:} 直线斜截式方程

\textbf{English Description:} Slope-intercept form of a straight line

\vspace{0.5cm}

\item \textbf{公式 111 (Formula 111)}

\begin{equation}
y - y_1 = m(x - x_1)
\end{equation}

\textbf{中文说明:} 直线点斜式方程

\textbf{English Description:} Point-slope form of a straight line

\vspace{0.5cm}

\item \textbf{公式 112 (Formula 112)}

\begin{equation}
Ax + By + C = 0
\end{equation}

\textbf{中文说明:} 直线的一般式方程

\textbf{English Description:} General equation of a straight line

\vspace{0.5cm}

\item \textbf{公式 113 (Formula 113)}

\begin{equation}
(x-h)^2 + (y-k)^2 = r^2
\end{equation}

\textbf{中文说明:} 已知圆心半径的圆的方程

\textbf{English Description:} Equation of a circle with center (h, k) and radius r

\vspace{0.5cm}

\item \textbf{公式 114 (Formula 114)}

\begin{equation}
\frac{x^2}{a^2} + \frac{y^2}{b^2} = 1
\end{equation}

\textbf{中文说明:} 椭圆标准方程

\textbf{English Description:} Standard equation of an ellipse

\vspace{0.5cm}

\item \textbf{公式 115 (Formula 115)}

\begin{equation}
\frac{x^2}{a^2} - \frac{y^2}{b^2} = 1
\end{equation}

\textbf{中文说明:} 双曲线标准方程

\textbf{English Description:} Standard equation of a hyperbola

\vspace{0.5cm}

\item \textbf{公式 116 (Formula 116)}

\begin{equation}
y^2 = 4ax
\end{equation}

\textbf{中文说明:} 抛物线标准方程

\textbf{English Description:} Standard equation of a parabola

\vspace{0.5cm}

\item \textbf{公式 117 (Formula 117)}

\begin{equation}
\sum_{k=0}^n \binom{n}{k} = 2^n
\end{equation}

\textbf{中文说明:} 二项式系数和

\textbf{English Description:} Sum of binomial coefficients

\vspace{0.5cm}

\item \textbf{公式 118 (Formula 118)}

\begin{equation}
(a+b)^n = \sum_{k=0}^n \binom{n}{k} a^{n-k} b^k
\end{equation}

\textbf{中文说明:} 二项式定理

\textbf{English Description:} Binomial theorem

\vspace{0.5cm}

\item \textbf{公式 119 (Formula 119)}

\begin{equation}
P(A \cap B) = P(A)P(B|A)
\end{equation}

\textbf{中文说明:} 概率乘法公式

\textbf{English Description:} Multiplication rule for probability

\vspace{0.5cm}

\item \textbf{公式 120 (Formula 120)}

\begin{equation}
P(A \cup B) = P(A) + P(B) - P(A \cap B)
\end{equation}

\textbf{中文说明:} 概率加法公式

\textbf{English Description:} Addition rule for probability

\vspace{0.5cm}

\item \textbf{公式 121 (Formula 121)}

\begin{equation}
P(A|B) = \frac{P(A \cap B)}{P(B)}
\end{equation}

\textbf{中文说明:} 条件概率公式

\textbf{English Description:} Conditional probability formula

\vspace{0.5cm}

\item \textbf{公式 122 (Formula 122)}

\begin{equation}
P(A') = 1 - P(A)
\end{equation}

\textbf{中文说明:} 对立事件的概率

\textbf{English Description:} Probability of the complement event

\vspace{0.5cm}

\item \textbf{公式 123 (Formula 123)}

\begin{equation}
E[X] = \sum_{i} x_i P(x_i)
\end{equation}

\textbf{中文说明:} 离散型随机变量的期望

\textbf{English Description:} Expected value of a discrete random variable

\vspace{0.5cm}

\item \textbf{公式 124 (Formula 124)}

\begin{equation}
\text{Var}(X) = E[X^2] - (E[X])^2
\end{equation}

\textbf{中文说明:} 随机变量的方差

\textbf{English Description:} Variance of a random variable

\vspace{0.5cm}

\item \textbf{公式 125 (Formula 125)}

\begin{equation}
f(x) = \frac{1}{\sigma\sqrt{2\pi}} e^{-\frac{(x-\mu)^2}{2\sigma^2}}
\end{equation}

\textbf{中文说明:} 正态分布概率密度函数

\textbf{English Description:} Probability density function of the normal distribution

\vspace{0.5cm}

\item \textbf{公式 126 (Formula 126)}

\begin{equation}
S = \frac{1}{n-1} \sum_{i=1}^n (x_i - \bar{x})^2
\end{equation}

\textbf{中文说明:} 样本方差公式

\textbf{English Description:} Sample variance formula

\vspace{0.5cm}

\item \textbf{公式 127 (Formula 127)}

\begin{equation}
\frac{d}{dx}f(g(x)) = f'(g(x))g'(x)
\end{equation}

\textbf{中文说明:} 导数的链式法则

\textbf{English Description:} Chain rule for derivatives

\vspace{0.5cm}

\item \textbf{公式 128 (Formula 128)}

\begin{equation}
\frac{d}{dx}(uv) = u'v + uv'
\end{equation}

\textbf{中文说明:} 导数的乘积法则

\textbf{English Description:} Product rule for derivatives

\vspace{0.5cm}

\item \textbf{公式 129 (Formula 129)}

\begin{equation}
\frac{d}{dx}\left(\frac{u}{v}\right) = \frac{u'v - uv'}{v^2}
\end{equation}

\textbf{中文说明:} 导数的商法则

\textbf{English Description:} Quotient rule for derivatives

\vspace{0.5cm}

\item \textbf{公式 130 (Formula 130)}

\begin{equation}
\int_{a}^{b} f(x) dx \approx \sum_{i=1}^{n} f(x_i) \Delta x
\end{equation}

\textbf{中文说明:} 数值积分(黎曼和)

\textbf{English Description:} Numerical integration (Riemann sum)

\vspace{0.5cm}

\item \textbf{公式 131 (Formula 131)}

\begin{equation}
|A \cup B| = |A| + |B| - |A \cap B|
\end{equation}

\textbf{中文说明:} 两个集合的容斥原理

\textbf{English Description:} Inclusion-exclusion principle for two sets

\vspace{0.5cm}

\item \textbf{公式 132 (Formula 132)}

\begin{equation}
(a+b+c)^2 = a^2 + b^2 + c^2 + 2ab + 2ac + 2bc
\end{equation}

\textbf{中文说明:} 三项式的平方公式

\textbf{English Description:} Square of a trinomial

\vspace{0.5cm}

\item \textbf{公式 133 (Formula 133)}

\begin{equation}
\frac{a_1}{b_1} = \frac{a_2}{b_2} = \cdots = \frac{a_n}{b_n}
\end{equation}

\textbf{中文说明:} 比例的通式

\textbf{English Description:} General form of a proportion

\vspace{0.5cm}

\item \textbf{公式 134 (Formula 134)}

\begin{equation}
\lim_{n \to \infty} \frac{1}{n} = 0
\end{equation}

\textbf{中文说明:} 1/n在n趋于无穷时的极限

\textbf{English Description:} Limit of 1/n as n approaches infinity

\vspace{0.5cm}

\item \textbf{公式 135 (Formula 135)}

\begin{equation}
A \subseteq B \iff \forall x (x \in A \implies x \in B)
\end{equation}

\textbf{中文说明:} A是B的子集,当且仅当A的每个元素都是B的元素。

\textbf{English Description:} A is a subset of B if and only if every element of A is also an element of B.

\vspace{0.5cm}

\item \textbf{公式 136 (Formula 136)}

\begin{equation}
A \cup B = \{x \mid x \in A \text{ or } x \in B\}
\end{equation}

\textbf{中文说明:} A与B的并集是所有属于A或B的元素组成的集合。

\textbf{English Description:} The union of A and B is the set of elements that are in A or in B.

\vspace{0.5cm}

\item \textbf{公式 137 (Formula 137)}

\begin{equation}
A \cap B = \{x \mid x \in A \text{ and } x \in B\}
\end{equation}

\textbf{中文说明:} A与B的交集是所有既属于A又属于B的元素组成的集合。

\textbf{English Description:} The intersection of A and B is the set of elements that are in both A and B.

\vspace{0.5cm}

\item \textbf{公式 138 (Formula 138)}

\begin{equation}
A - B = \{x \mid x \in A \text{ and } x \notin B\}
\end{equation}

\textbf{中文说明:} A与B的差集是所有属于A但不属于B的元素组成的集合。

\textbf{English Description:} The difference of A and B is the set of elements that are in A but not in B.

\vspace{0.5cm}

\item \textbf{公式 139 (Formula 139)}

\begin{equation}
A^c = U - A
\end{equation}

\textbf{中文说明:} A的补集是全集U中所有不属于A的元素组成的集合。

\textbf{English Description:} The complement of A is the set of elements in the universal set U that are not in A.

\vspace{0.5cm}

\item \textbf{公式 140 (Formula 140)}

\begin{equation}
(A \cup B)^c = A^c \cap B^c
\end{equation}

\textbf{中文说明:} 德摩根律:A和B的并集的补集等于A的补集与B的补集的交集。

\textbf{English Description:} De Morgan's law: The complement of the union of A and B is the intersection of their complements.

\vspace{0.5cm}

\item \textbf{公式 141 (Formula 141)}

\begin{equation}
(A \cap B)^c = A^c \cup B^c
\end{equation}

\textbf{中文说明:} 德摩根律:A和B的交集的补集等于A的补集与B的补集的并集。

\textbf{English Description:} De Morgan's law: The complement of the intersection of A and B is the union of their complements.

\vspace{0.5cm}

\item \textbf{公式 142 (Formula 142)}

\begin{equation}
A \cup \emptyset = A
\end{equation}

\textbf{中文说明:} 任意集合与空集的并集仍为该集合本身。

\textbf{English Description:} The union of any set and the empty set is the set itself.

\vspace{0.5cm}

\item \textbf{公式 143 (Formula 143)}

\begin{equation}
A \cap \emptyset = \emptyset
\end{equation}

\textbf{中文说明:} 任意集合与空集的交集仍为空集。

\textbf{English Description:} The intersection of any set and the empty set is the empty set.

\vspace{0.5cm}

\item \textbf{公式 144 (Formula 144)}

\begin{equation}
A \cup A = A
\end{equation}

\textbf{中文说明:} 集合与自身的并集等于自身。

\textbf{English Description:} The union of a set with itself is itself.

\vspace{0.5cm}

\item \textbf{公式 145 (Formula 145)}

\begin{equation}
A \cap A = A
\end{equation}

\textbf{中文说明:} 集合与自身的交集等于自身。

\textbf{English Description:} The intersection of a set with itself is itself.

\vspace{0.5cm}

\item \textbf{公式 146 (Formula 146)}

\begin{equation}
A \cup B = B \cup A
\end{equation}

\textbf{中文说明:} 并集满足交换律。

\textbf{English Description:} Union is commutative.

\vspace{0.5cm}

\item \textbf{公式 147 (Formula 147)}

\begin{equation}
A \cap B = B \cap A
\end{equation}

\textbf{中文说明:} 交集满足交换律。

\textbf{English Description:} Intersection is commutative.

\vspace{0.5cm}

\item \textbf{公式 148 (Formula 148)}

\begin{equation}
(A \cup B) \cup C = A \cup (B \cup C)
\end{equation}

\textbf{中文说明:} 并集满足结合律。

\textbf{English Description:} Union is associative.

\vspace{0.5cm}

\item \textbf{公式 149 (Formula 149)}

\begin{equation}
(A \cap B) \cap C = A \cap (B \cap C)
\end{equation}

\textbf{中文说明:} 交集满足结合律。

\textbf{English Description:} Intersection is associative.

\vspace{0.5cm}

\item \textbf{公式 150 (Formula 150)}

\begin{equation}
A \cap (B \cup C) = (A \cap B) \cup (A \cap C)
\end{equation}

\textbf{中文说明:} 交集对并集满足分配律。

\textbf{English Description:} Intersection distributes over union.

\vspace{0.5cm}

\item \textbf{公式 151 (Formula 151)}

\begin{equation}
A \cup (B \cap C) = (A \cup B) \cap (A \cup C)
\end{equation}

\textbf{中文说明:} 并集对交集满足分配律。

\textbf{English Description:} Union distributes over intersection.

\vspace{0.5cm}

\item \textbf{公式 152 (Formula 152)}

\begin{equation}
P(A) = 2^{|A|}
\end{equation}

\textbf{中文说明:} A的幂集的基数是2的A的基数次方。

\textbf{English Description:} The cardinality of the power set of A is 2 to the power of the cardinality of A.

\vspace{0.5cm}

\item \textbf{公式 153 (Formula 153)}

\begin{equation}
|A \cup B| = |A| + |B| - |A \cap B|
\end{equation}

\textbf{中文说明:} 有限集合A与B并集的基数公式。

\textbf{English Description:} Cardinality of the union of two finite sets.

\vspace{0.5cm}

\item \textbf{公式 154 (Formula 154)}

\begin{equation}
f'(x) = \lim_{h \to 0} \frac{f(x+h) - f(x)}{h}
\end{equation}

\textbf{中文说明:} 导数的定义

\textbf{English Description:} Definition of the derivative

\vspace{0.5cm}

\item \textbf{公式 155 (Formula 155)}

\begin{equation}
|E| \leq \frac{n(n-1)}{2}
\end{equation}

\textbf{中文说明:} 一个有n个顶点的简单无向图最多有 n(n-1)/2 条边。

\textbf{English Description:} The maximum number of edges in a simple undirected graph with n vertices.

\vspace{0.5cm}

\item \textbf{公式 156 (Formula 156)}

\begin{equation}
\sum_{v \in V} deg(v) = 2|E|
\end{equation}

\textbf{中文说明:} 无向图中所有顶点的度之和等于边数的两倍。

\textbf{English Description:} The sum of the degrees of all vertices in an undirected graph equals twice the number of edges.

\vspace{0.5cm}

\item \textbf{公式 157 (Formula 157)}

\begin{equation}
q = a + bi + cj + dk
\end{equation}

\textbf{中文说明:} 四元数q由实部a和虚部b、c、d组成。

\textbf{English Description:} A quaternion q is composed of a real part a and three imaginary parts b, c, d.

\vspace{0.5cm}

\item \textbf{公式 158 (Formula 158)}

\begin{equation}
i^2 = j^2 = k^2 = ijk = -1
\end{equation}

\textbf{中文说明:} 四元数单位i, j, k的基本关系。

\textbf{English Description:} The fundamental relations of quaternion units i, j, k.

\vspace{0.5cm}

\item \textbf{公式 159 (Formula 159)}

\begin{equation}
q = \cos\left(\frac{\theta}{2}\right) + (xi + yj + zk)\sin\left(\frac{\theta}{2}\right)
\end{equation}

\textbf{中文说明:} 单位四元数表示绕单位向量(x, y, z)旋转$\theta$角。

\textbf{English Description:} A unit quaternion representing a rotation by $\theta$ around the unit vector (x, y, z).

\vspace{0.5cm}

\item \textbf{公式 160 (Formula 160)}

\begin{equation}
v' = qvq^{-1}
\end{equation}

\textbf{中文说明:} 用四元数q对向量v(视为纯四元数)进行旋转。

\textbf{English Description:} Rotation of vector v (as a pure quaternion) by quaternion q.

\vspace{0.5cm}

\item \textbf{公式 161 (Formula 161)}

\begin{equation}
C_n = \frac{n(n-1)}{2}
\end{equation}

\textbf{中文说明:} n阶完全图的边数为 n(n-1)/2。

\textbf{English Description:} The number of edges in a complete graph with n vertices.

\vspace{0.5cm}

\item \textbf{公式 162 (Formula 162)}

\begin{equation}
A^k_{ij} = \text{number of walks of length k from vertex i to j}
\end{equation}

\textbf{中文说明:} 邻接矩阵的k次幂的(i,j)元素表示从顶点i到顶点j长度为k的步数。

\textbf{English Description:} The (i,j) entry of the k-th power of the adjacency matrix gives the number of walks of length k from vertex i to j.

\vspace{0.5cm}

\item \textbf{公式 163 (Formula 163)}

\begin{equation}
|V| - |E| + |F| = 2
\end{equation}

\textbf{中文说明:} 欧拉公式:平面图的顶点数减去边数加上面数等于2。

\textbf{English Description:} Euler's formula for planar graphs: vertices minus edges plus faces equals 2.

\vspace{0.5cm}

\item \textbf{公式 164 (Formula 164)}

\begin{equation}
a \mid b \iff \exists k \in \mathbb{Z},\ b = ak
\end{equation}

\textbf{中文说明:} a整除b当且仅当存在整数k使b=ak。

\textbf{English Description:} a divides b if and only if there exists an integer k such that b = ak.

\vspace{0.5cm}

\item \textbf{公式 165 (Formula 165)}

\begin{equation}
\gcd(a, b) = \max\{d \mid d \mid a \text{ and } d \mid b\}
\end{equation}

\textbf{中文说明:} 最大公约数定义。

\textbf{English Description:} Greatest common divisor of a and b.

\vspace{0.5cm}

\item \textbf{公式 166 (Formula 166)}

\begin{equation}
\text{If}\ \gcd(a, b) = 1,\ \exists x, y \in \mathbb{Z}\ \text{such that}\ ax + by = 1
\end{equation}

\textbf{中文说明:} 裴蜀等式(若a与b互素,则存在整数x,y使ax+by=1)。

\textbf{English Description:} Bezout's identity for coprime integers.

\vspace{0.5cm}

\item \textbf{公式 167 (Formula 167)}

\begin{equation}
\operatorname{lcm}(a, b) = \frac{ab}{\gcd(a, b)}
\end{equation}

\textbf{中文说明:} 最小公倍数公式。

\textbf{English Description:} Least common multiple of a and b.

\vspace{0.5cm}

\item \textbf{公式 168 (Formula 168)}

\begin{equation}
a \equiv b \pmod{n} \iff n \mid (a-b)
\end{equation}

\textbf{中文说明:} 模n同余的定义。

\textbf{English Description:} Definition of congruence modulo n.

\vspace{0.5cm}

\item \textbf{公式 169 (Formula 169)}

\begin{equation}
\phi(n) = |\{1 \leq k \leq n\mid \gcd(k,n) = 1\}|
\end{equation}

\textbf{中文说明:} 欧拉函数定义。

\textbf{English Description:} Euler's totient function.

\vspace{0.5cm}

\item \textbf{公式 170 (Formula 170)}

\begin{equation}
a^{\phi(n)} \equiv 1 \pmod{n}\ (\gcd(a,n)=1)
\end{equation}

\textbf{中文说明:} 欧拉定理。

\textbf{English Description:} Euler's theorem.

\vspace{0.5cm}

\item \textbf{公式 171 (Formula 171)}

\begin{equation}
a^p \equiv a \pmod{p}
\end{equation}

\textbf{中文说明:} 费马小定理(p为素数)。

\textbf{English Description:} Fermat's little theorem (p is prime).

\vspace{0.5cm}

\item \textbf{公式 172 (Formula 172)}

\begin{equation}
\sum_{d|n} \phi(d) = n
\end{equation}

\textbf{中文说明:} 欧拉函数在所有正约数上的和等于n。

\textbf{English Description:} Sum of Euler's totient function over all divisors equals n.

\vspace{0.5cm}

\item \textbf{公式 173 (Formula 173)}

\begin{equation}
\mu(n) = \begin{cases} 1 & n=1 \\ 0 & \text{if } n \text{ has a squared prime factor} \\ (-1)^k & n \text{ is a product of } k \text{ distinct primes} \end{cases}
\end{equation}

\textbf{中文说明:} 莫比乌斯函数的定义。

\textbf{English Description:} Definition of the Möbius function.

\vspace{0.5cm}

\item \textbf{公式 174 (Formula 174)}

\begin{equation}
\sigma(n) = \sum_{d|n} d
\end{equation}

\textbf{中文说明:} 约数和函数。

\textbf{English Description:} Sum of divisors function.

\vspace{0.5cm}

\item \textbf{公式 175 (Formula 175)}

\begin{equation}
\pi(x) \sim \frac{x}{\log x}
\end{equation}

\textbf{中文说明:} 素数定理,素数分布的渐近公式。

\textbf{English Description:} Prime number theorem (asymptotic estimate for the number of primes $\leq$ x).

\vspace{0.5cm}

\item \textbf{公式 176 (Formula 176)}

\begin{equation}
f(x+y) = f(x) + f(y)
\end{equation}

\textbf{中文说明:} 阿贝尔-柯西方程。

\textbf{English Description:} Cauchy equation for additive functions.

\vspace{0.5cm}

\item \textbf{公式 177 (Formula 177)}

\begin{equation}
B(x_0, r) = \{ x \in X \mid d(x, x_0) < r \}
\end{equation}

\textbf{中文说明:} 度量空间中的开球。

\textbf{English Description:} Open ball in a metric space.

\vspace{0.5cm}

\item \textbf{公式 178 (Formula 178)}

\begin{equation}
U \text{ is open } \iff \forall x \in U, \exists \varepsilon > 0, B(x, \varepsilon) \subseteq U
\end{equation}

\textbf{中文说明:} 度量空间中开集的定义。

\textbf{English Description:} Definition of open set in a metric space.

\vspace{0.5cm}

\item \textbf{公式 179 (Formula 179)}

\begin{equation}
\overline{A} = \bigcap_{F \supseteq A, F \text{ closed}} F
\end{equation}

\textbf{中文说明:} 集合A的闭包。

\textbf{English Description:} Closure of a set: intersection of all closed supersets.

\vspace{0.5cm}

\item \textbf{公式 180 (Formula 180)}

\begin{equation}
A \text{ is closed } \iff X \setminus A \text{ is open}
\end{equation}

\textbf{中文说明:} 集合闭集的定义。

\textbf{English Description:} A set is closed if and only if its complement is open.

\vspace{0.5cm}

\item \textbf{公式 181 (Formula 181)}

\begin{equation}
f \text{ is continuous} \iff \forall U \subseteq Y \text{ open}, f^{-1}(U) \text{ is open in } X
\end{equation}

\textbf{中文说明:} 连续映射的拓扑定义。

\textbf{English Description:} Topological definition of continuity.

\vspace{0.5cm}

\item \textbf{公式 182 (Formula 182)}

\begin{equation}
\chi = V - E + F
\end{equation}

\textbf{中文说明:} 欧拉示性数。

\textbf{English Description:} Euler characteristic for polyhedra (V: vertices, E: edges, F: faces).

\vspace{0.5cm}

\item \textbf{公式 183 (Formula 183)}

\begin{equation}
\bigcup_{\alpha \in I} U_\alpha \text{ is open if every } U_\alpha \text{ is open}
\end{equation}

\textbf{中文说明:} 任意多个开集的并仍为开集。

\textbf{English Description:} Arbitrary unions of open sets are open.

\vspace{0.5cm}

\item \textbf{公式 184 (Formula 184)}

\begin{equation}
\bigcap_{i=1}^n U_i \text{ is open if every } U_i \text{ is open}
\end{equation}

\textbf{中文说明:} 有限多个开集的交仍为开集。

\textbf{English Description:} Finite intersections of open sets are open.

\vspace{0.5cm}

\item \textbf{公式 185 (Formula 185)}

\begin{equation}
\frac{1}{\pi} = \frac{2\sqrt{2}}{9801} \sum_{k=0}^{\infty} \frac{(4k)! (1103 + 26390k)}{(k!)^4 396^{4k}}
\end{equation}

\textbf{中文说明:} 拉马努金关于1/$\pi$的级数,用于高精度计算$\pi$。

\textbf{English Description:} Ramanujan's remarkable series for 1/$\pi$, used in high-precision calculations of $\pi$.

\vspace{0.5cm}

\item \textbf{公式 186 (Formula 186)}

\begin{equation}
\sum_{n=1}^{\infty} \frac{1}{n^2} = \frac{\pi^2}{6}
\end{equation}

\textbf{中文说明:} 巴塞尔问题的解,被拉马努金重新发现并扩展。

\textbf{English Description:} Basel problem solution, rediscovered and expanded by Ramanujan.

\vspace{0.5cm}

\item \textbf{公式 187 (Formula 187)}

\begin{equation}
\sum_{n=0}^{\infty} \frac{(\frac{1}{2})_n^3}{n!^3}(6n+1)\frac{1}{4^n} = \frac{4}{\pi}
\end{equation}

\textbf{中文说明:} 拉马努金提出的又一个1/$\pi$级数。

\textbf{English Description:} Another Ramanujan series for 1/$\pi$.

\vspace{0.5cm}

\item \textbf{公式 188 (Formula 188)}

\begin{equation}
e^{\pi \sqrt{163}} \approx 262537412640768743.99999999999925
\end{equation}

\textbf{中文说明:} 拉马努金的近似整数,模函数中的著名“近整数”。

\textbf{English Description:} Ramanujan's almost integer, a famous example of an 'almost integer' from modular functions.

\vspace{0.5cm}

\item \textbf{公式 189 (Formula 189)}

\begin{equation}
\frac{1}{\pi} = \frac{2\sqrt{2}}{99^2} \sum_{k=0}^{\infty} \frac{(4k)! (21460k+1123)}{(k!)^4 396^{4k}}
\end{equation}

\textbf{中文说明:} 拉马努金关于1/$\pi$的另一种级数。

\textbf{English Description:} A variant of Ramanujan's series for 1/$\pi$.

\vspace{0.5cm}

\item \textbf{公式 190 (Formula 190)}

\begin{equation}
1729 = 1^3 + 12^3 = 9^3 + 10^3
\end{equation}

\textbf{中文说明:} 著名的“出租车数”,由拉马努金提出。

\textbf{English Description:} The famous 'taxicab number', highlighted by Ramanujan.

\vspace{0.5cm}

\item \textbf{公式 191 (Formula 191)}

\begin{equation}
\sigma(n) = \sum_{d|n} d
\end{equation}

\textbf{中文说明:} 拉马努金对因子函数的研究,$\sigma$(n)表示n的所有因子的和。

\textbf{English Description:} Ramanujan's study of the divisor function, where $\sigma$(n) is the sum of the divisors of n.

\vspace{0.5cm}

\item \textbf{公式 192 (Formula 192)}

\begin{equation}
p(n) \sim \frac{1}{4n\sqrt{3}} \exp\left(\pi\sqrt{\frac{2n}{3}}\right)
\end{equation}

\textbf{中文说明:} 拉马努金关于分割函数p(n)的渐近公式。

\textbf{English Description:} Ramanujan's asymptotic formula for the partition function p(n).

\vspace{0.5cm}

\item \textbf{公式 193 (Formula 193)}

\begin{equation}
\sum_{n=1}^{\infty} \frac{n}{e^{2\pi n}-1} = \frac{1}{24} - \frac{1}{8\pi}
\end{equation}

\textbf{中文说明:} 拉马努金关于指数与$\pi$的级数。

\textbf{English Description:} A Ramanujan series involving exponential and $\pi$.

\vspace{0.5cm}

\item \textbf{公式 194 (Formula 194)}

\begin{equation}
\zeta(s) = 2^s \pi^{s-1} \sin\left(\frac{\pi s}{2}\right) \Gamma(1-s) \zeta(1-s)
\end{equation}

\textbf{中文说明:} 黎曼$\zeta$函数的函数方程,拉马努金曾深入研究。

\textbf{English Description:} Functional equation of the Riemann zeta function, studied by Ramanujan.


\end{enumerate}

\end{document}
